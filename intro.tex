\documentclass[runningheads]{llncs}
\usepackage{amsfonts}
\usepackage{amsmath}

\newtheorem{mytheorem}{Theorem}
\newtheorem{mylemma}[mytheorem]{Lemma}
\newtheorem{myquestion}[mytheorem]{Question}
\newtheorem{myexample}[mytheorem]{Example}
\newtheorem{myproposition}[mytheorem]{Proposition}
\newtheorem{mycorollary}[mytheorem]{Corollary}
\newtheorem{mydefinition}[mytheorem]{Definition}
\newtheorem{myprinciple}[mytheorem]{Principle}

\begin{document}

\title{Extending environments to incentivize self-reflection
in reinforcement learning}
\titlerunning{Extending environments to incentivize self-reflection}

\author{Samuel Allen Alexander\inst{1}\orcidID{0000-0002-7930-110X}}
\authorrunning{S.\ A.\ Alexander}
\institute{The U.S.\ Securities and Exchange Commission
\email{samuelallenalexander@gmail.com}
\url{https://philpeople.org/profiles/samuel-alexander/publications}}

\maketitle

\begin{abstract}
    We consider an extended notion
    of reinforcement learning environment, in which the environment is able
    to simulate the agent. We give
    examples of some such extended environments which seem to incentivize
    various different types of self-reflection or self-awareness.
\end{abstract}

\section{Introduction}

In this paper, we consider computable agents in a deterministic variant of the reinforcement
learning (RL) framework (we restrict our attention to deterministic environments and
agents for the sake of simplicity, but the basic ideas would not be difficult to adapt
to non-deterministic RL). Such agents are intended to interact with deterministic RL
environments, where they take actions and receive rewards and observations in
response to those actions.

Traditionally, a deterministic RL environment is essentially a function
$e$ which outputs a reward-observation pair
$\langle r,o\rangle=e(r_0,o_0,a_0,\ldots,r_n,o_n,a_n)$ in response to an
award-observation-action sequence $(a_0,\ldots,a_n)$.
However, there is another type of environment in which
computable RL agents can interact just as well. Essentially, we define an \emph{extended
environment} to be a function $e$ which outputs a reward-observation pair
$\langle r,o\rangle=e(T,\langle r_0,o_0,a_0,\ldots,r_n,o_n,a_n \rangle)$ in response
to a reward-observation-action sequence
along with a Turing machine $T$ which computes the agent.
Intuitively, this should be thought of as follows: when the agent enters the environment,
the environment is made aware of the agent's source-code, and can use that
source-code to simulate the agent when computing rewards and observations.

For example, imagine a game\footnote{This game bears some similarity to Newcomb's
Paradox \cite{nozick1969newcomb}.} consisting of a
labyrinth where the player wanders from
room to room, each room containing a treasure, and most (but not all) rooms containing
a guard.
\begin{itemize}
    \item
    The player can only see one room at a time, and cannot look ahead into adjacent
    rooms, nor visit the same room twice.
    \item
    In a room with no guard, the player can take the treasure, yielding a reward.
    \item
    If the player chooses to take the treasure in a guarded room,
    then, by simulating the player, the guard determines: ``If
    this room had been unguarded, would the player have taken the treasure?''
    If so, the guard blocks the player and zaps them (yielding a negative reward).
    Otherwise, the guard allows the player to take the treasure (and the player is rewarded).
\end{itemize}

The above game would apparently be impossible (or prohibitively expensive)
for a human to play, due to the difficulty of simulating a human player.
But there is no reason why the above game could not be played by an RL agent (the
agent's source-code being given to the game-engine beforehand). Clearly this kind of
extended environment is not the kind of environment the RL agent is intended to be
applied to and is not representative of the kind of environments RL agents are
applied to\footnote{Such environments might, however, accidentally arise if both environment
and agent are implemented on the same machine and the environment is managed by an AI
sophisticated enough to exploit unintended informational side channels, as in
\cite{yampolskiy2012leakproofing}.}. However, such environments could be
useful on the path to Artificial
General Intelligence (AGI) because they seem to incentivize self-reflection: in order
to perform well across many games like the above, the player would evidently need
some degree of self-awareness.

We will give examples of interesting
extended environments of the above kind, designed to incentivize RL agents to
recursively engage in self-reflection in various ways.
We conjecture that traditional RL agents would perform poorly in these extended
environments, because traditional RL techniques do not involve any sort of
self-reflection. We hope these examples will facilitate
new RL techniques that do involve self-reflection, hopefully as a step toward AGI.

\section{Preliminaries}

\begin{definition}
(Plays and prompts)
    \begin{enumerate}
        \item
        By an \emph{ROA-play}, we mean either the empty sequence $\langle\rangle$,
        or else a sequence of the form
        \[
            \langle r_0,o_0,a_0,\ldots,r_k,o_k,a_k\rangle
        \]
        where each $r_i\in\mathbb Q$ (thought of as a \emph{reward}),
        each $o_i\in\mathbb N$ (thought of as an \emph{observation}),
        and each $a_k\in \mathbb N$ (thought of as an \emph{action}).
        \item
        By an \emph{ROA-prompt}, we mean a sequence of the form
        $s\frown r\frown o$
        where $s$ is an ROA-play, $\frown$ denotes concatenation,
        $r\in\mathbb Q$ (thought of as a \emph{reward}),
        and $o\in\mathbb N$ (thought of as an \emph{observation}).
    \end{enumerate}
\end{definition}

\begin{lemma}
\label{roaplaydecompositionlemma}
    If $s$ is an ROA-play, then either $s=\langle\rangle$, or else
    $s=p\frown a$ for some ROA-prompt $p$ and action $a\in\mathbb N$.
\end{lemma}

\begin{proof}
    If $s\not=\langle\rangle$ then we can write
    $s=\langle r_0,o_0,a_0,\ldots,r_k,o_k,a_k \rangle$.

    \textbf{Case 1:}
    $k=0$. By definition, $\langle\rangle$ is an ROA-play,
    therefore $p=\langle\rangle\frown r_0\frown o_0$ is an ROA-prompt,
    and $s=p\frown a_0$.

    \textbf{Case 2:} $k>0$.
    Then $p'=\langle r_0,o_0,a_0,\ldots,r_{k-1},o_{k-1},a_{k-1}\rangle$ is an ROA-play,
    therefore $p=p'\frown r_k\frown o_k$ is an ROA-prompt,
    and $s=p\frown a_k$.
\end{proof}

\begin{definition}
\label{agentandenvironment}
(Agents and environments)
    \begin{enumerate}
    \item An \emph{agent} is a Turing machine which
    halts whenever it is run on an ROA-prompt, outputting
    an action $a\in\mathbb N$.
    \item An \emph{extended environment} is a Turing machine $e$ such that:
        \begin{itemize}
            \item
            For every agent $T$, for every
            ROA-play $s$,
            when $e$ is run on input $\langle T,s\rangle$, $e$ halts on that input,
            outputting a pair $\langle r,o\rangle$ where $r\in\mathbb Q$ (thought of
            as a reward) and $o\in\mathbb N$ (thought of as an observation).
        \end{itemize}
    \end{enumerate}
\end{definition}

There is a subtle nuance in Definition \ref{agentandenvironment}. Should the agent's
next action depend on the entire history (including prior actions), or only on prior
rewards and observations? One could argue that
the agent's next action needn't depend on its own past actions, since its own past actions
can be inferred from past rewards and observations.
Normally, it would not matter much whether
or not the agent's next action depend on its own past actions\footnote{In
\cite{alexander2019intelligence}, we
formalize agents whose actions depend only on past rewards and observations.
Legg and Hutter give a formalization where the agent's next action does explicitly
depend on its past actions \cite{legg2007universal}.}. In formalizing examples of extended
environments that incentivize self-reflection, we have found it convenient for the agent's
next action to formally depend on past actions. Perhaps this reflects that known
conscious agents (e.g.\ humans)
evidently do \emph{not} carefully re-compute their own
past actions from remembered observations and rewards, but instead, a human maintains
memories of her past actions as well, regardless whether doing so is formally superfluous.


\begin{definition}
\label{interactiondefn}
    Suppose $T$ is an agent and $e$ is an extended environment.
    The \emph{result of $T$ interacting with $e$} is the infinite
    reward-observation-action sequence
    \[\langle r_0,o_0,a_0,r_1,o_1,a_1,\ldots\rangle\]
    (each $r_i\in\mathbb Q$, $o_i,a_i\in\mathbb N$)
    defined inductively as follows.
    \begin{itemize}
        \item $r_0$ and $o_0$ are obtained by computing $e$ on
        $\langle T,\langle\rangle\rangle$.
        \item $a_0$ is the output of $T$ on $\langle r_0,o_0\rangle$.
        \item For $i>0$, $r_i$ and $o_i$ are obtained by computing $e$
        on
        \[\langle T,\langle r_0,o_0,a_0,\ldots,r_{i-1},o_{i-1},a_{i-1}\rangle\rangle.\]
        \item For $i>0$, $a_i$ is obtained by computing $T$ on
        \[\langle r_0,o_0,a_0,\ldots,r_{i-1},o_{i-1},a_{i-1},r_i,o_i\rangle.\]
    \end{itemize}
\end{definition}

\begin{lemma}
    For any agent $T$ and extended environment $e$, the result of $T$ interacting
    with $e$ (Definition \ref{interactiondefn}) is defined (all of the computations
    in question halt with the necessary outputs).
\end{lemma}

\begin{proof}
    By a simultaneous induction:
    \begin{itemize}
        \item
        Each $r_i$ and $o_i$ are defined (and $r_i\in\mathbb Q$
        and $o_i\in\mathbb Q$) because, by induction,
        $\langle r_0,o_0,a_0,\ldots,r_{i-1},o_{i-1},a_{i-1}\rangle$
        is defined and is an ROA-play (because, inductively,
        each $r_j\in\mathbb Q$, $o_j\in\mathbb N$ and $a_j\in\mathbb N$
        for all $j<i$) and thus
        $r_i$ and $o_i$ are defined with the correct form by
        Definition \ref{agentandenvironment} (part 2).
        \item
        Each $a_i$ is defined (and $a_i\in\mathbb N$) because, by induction,
        $\langle r_0,o_0,a_0,\ldots,r_i,o_i\rangle$
        is defined and is an ROA-prompt (similar to the above) and thus
        $a_i$ is defined with the correct form by Definition
        \ref{agentandenvironment} (part 1).
    \end{itemize}
\end{proof}

One important implication of extended environments is that they further divide
the (already divided) ways of measuring intelligence of RL agents. Intelligence
measures
\cite{alexander2019intelligence} \cite{hernandez} \cite{legg2007universal}
which aggregate performance over traditional environments only measure
an agent's intelligence over those environments. The same measures could easily
be extended to also take extended environments into account, perhaps providing
measures which better capture agents' self-reflection abilities.

\section{Examples of Self-reflection-incentivizing Environments}

In this section, we give examples of some interesting extended environments which seem
to incentivize various forms of self-reflection. We are inspired by various libraries of
traditional RL environments \cite{bellemare2013arcade}
\cite{beyret2019animal} \cite{brockman2016openai} \cite{chollet2019measure}
\cite{cobbe2020leveraging}. Some of our examples do not provide individual environments,
but rather, procedures for obtaining new environments from old (these could be iterated,
in order to build up environments incentivizing deeper and deeper nested self-reflection).

\begin{example}
\label{rewardagentforignoringrewardsexample}
    (Reward Agent for Ignoring Rewards)
    For each ROA-prompt $p$, let $p^0$ be the ROA-prompt equal to $p$ except that
    all rewards are $0$.
    For any environment $e$, we define a new environment
    $e'$ as follows
    (where $T$ is a Turing machine, $p$ is an ROA-prompt, and $a\in \mathbb N$ is thought of as
    the agent's action in response to $p$):
    \begin{align*}
        e'(T,\langle\rangle) &= e(T,\langle\rangle)\\
        e'(T,p\frown a)
        &= \langle r,o\rangle,
    \end{align*}
    where $o$ is the observation component of
    $e(T,p\frown a)$ (if $e$ does not halt on input $\langle T,p\frown a\rangle$
    (resp.\ $\langle T,\langle\rangle\rangle$)
    then neither does $e'$) and
    \[
        r =
        \begin{cases}
            1 & \mbox{if $a=T(p^0)$,}\\
            -1 & \mbox{if $a\not=T(p^0)$}
        \end{cases}
    \]
    (if $T$ does not halt on $p^0$ then $e'$ does not halt on
    $\langle T,p\frown a\rangle$).
\end{example}

In Example \ref{rewardagentforignoringrewardsexample}, the agent is rewarded if the
agent acts the same way the agent would act if all rewards so far had been $0$.
Otherwise, the agent is punished. Thus, paradoxically, the agent is rewarded for
ignoring rewards. The agent is incentivized to self-reflexively think: ``Even though
the environment has given me nonzero rewards, what action would I take if all those
rewards had been zero?''

\begin{lemma}
\label{example1workslemma}
    Example \ref{rewardagentforignoringrewardsexample} really does define an
    extended environment.
\end{lemma}

\begin{proof}
    Fix an environment $e$ and let $e'$ be as in
    Example \ref{rewardagentforignoringrewardsexample}.
    We must show $e'$ is an extended environment (Definition \ref{agentandenvironment}
    part 2). We must show that for each agent $T$ and ROA-play $s$,
    $e'$ halts on $\langle T,s\rangle$ and outputs a pair $\langle r,o\rangle$
    such that $r\in\mathbb Q$ and $o\in\mathbb N$.

    \textbf{Case 1:} $s=\langle\rangle$. Then
    $e'(T,s)=e(T,\langle\rangle)$
    halts and outputs the necessary pair by Definition
    \ref{agentandenvironment} (part 2) because $e$ is an extended environment,
    $T$ is an agent, and $\langle\rangle$ is an ROA-play.

    \textbf{Case 2:} $s\not=\langle\rangle$. By Lemma \ref{roaplaydecompositionlemma},
    $s=p\frown a$ for some ROA-prompt $p$ and action $a\in\mathbb N$.
    Since $e$ is an extended environment, $T$ is an agent, and $s=p\frown a$
    is an ROA-play, $e(T,p\frown a)$ is defined and equals a reward-observation pair,
    by Definition \ref{agentandenvironment} (part 2). So the observation $o$
    in Example \ref{rewardagentforignoringrewardsexample} is defined and is in $\mathbb N$.
    Since $p$ is an ROA-prompt, clearly $p^0$ is also an ROA-prompt, therefore
    since $T$ is an agent, Definition \ref{agentandenvironment} (part 1)
    guarantees $T(p^0)$ is defined and is in $\mathbb N$. It follows that the reward
    $r$ in Definition \ref{rewardagentforignoringrewardsexample} is defined and is in
    $\mathbb Q$.
\end{proof}

For future examples, we will suppress the corresponding lemmas like
Lemma \ref{example1workslemma} which say that those examples really work.

Example \ref{rewardagentforignoringrewardsexample} is profound because it illustrates how,
in an extended environment, it is possible to give one sequence of rewards
in order to incentivize the agent to act as if a different sequence of rewards was given.
Imagine hiring a contractor and instructing him or her:
\begin{quote}
    I want to impress some clients. Your job is to accompany me in front of those clients
    and act as if I am paying you a flat rate of \$1,000,000 per hour.
    Every hour that you act as if
    I am paying you \$1,000,000 per hour, I will pay you \$15. But every hour that
    you do \emph{not} act as if I am paying you \$1,000,000 per hour,
    I will take \$15 out of your paycheck.
\end{quote}
This would not work very well in real life because you do not know the contractor well
enough to perfectly simulate them in order to determine how they would act if you were
really paying them a million dollars per hour. You would get in a fight, saying, ``I don't
think you acted like you were being paid \$1M per hour,'' and they would fight back,
saying, ``You're wrong, that \emph{is} how I would act if I were being paid \$1M per
hour.'' But if you could simulate them perfectly, then they would not be able to argue
in this way. Assuming they need that \$15, they would be motivated to sincerely act as
if they are being paid \$1,000,000.

\begin{example}
\label{selfinsertionexample}
    (Reward Agent for Self-Inserting)
    Fix a canonical computable bijection
    $o\mapsto \hat o$
    from $\mathbb N$ to $\mathbb Q\times \mathbb N$:
    thus, every observation $o$ encodes a reward-observation pair
    $\hat o = \langle r',o'\rangle$, and every reward-observation pair
    is encoded by some such $o$.
    For any environment $e$, we define
    a new environment $e'$ as follows
    (where $T,p,a$ are as in Example \ref{rewardagentforignoringrewardsexample},
    and with similar non-halting caveats
    as Example \ref{rewardagentforignoringrewardsexample}):
    \begin{align*}
        e'(T,\langle\rangle) &= e(T,\langle\rangle)\\
        e'(T,p\frown a) &= \langle r,o\rangle,
    \end{align*}
    where $o$ is such that $\hat o = e(T,p\frown a)$ and
    \[
        r =
        \begin{cases}
            1 & \mbox{if $a=T(p')$,}\\
            -1 & \mbox{if $a\not=T(p')$}
        \end{cases}
    \]
    where
    $p'$ is the ROA-prompt obtained from $p$
    by replacing each reward-observation pair
    $\ldots,r_i,o_i,\ldots$ by the reward-observation
    pair $\ldots,\widehat{o_i},\ldots$.
\end{example}

In Example \ref{selfinsertionexample}, one might imagine $e'$ as a room containing nothing
but an arcade game $e$. There is nothing for the agent in the room to do
except play this arcade game.
When played, the arcade game
visually displays rewards, but the agent merely observes them, and does not
``feel'' them. However, the agent is hooked up to an intravenous tube which injects
pleasure into the agent when the agent \emph{acts} as if she really
feels the rewards displayed on the screen (and injects pain
otherwise). In this way, the agent is incentivized
to self-identify with the protagonist in the video-game, self-reflexively asking,
``Which action would I take if those displayed rewards were real?''

\begin{example}
\label{incentivetoincentivizeexample}
    (Incentive to Incentivize)
    We define an environment $e$ as follows
    (where $T$ is a Turing machine and $r_0,o_0,a_0,\ldots,r_n,o_n,a_n$
    is an ROA-play,
    with similar non-halting caveats as in
    Example \ref{rewardagentforignoringrewardsexample}).
    \begin{align*}
        e(T,\langle\rangle) &= \langle 0, 0\rangle\\
        e(T,\langle r_0,o_0,a_0,\ldots,r_n,o_n,a_n\rangle) &= \langle r, 0\rangle,
    \end{align*}
    where
    \[
        r =
        \begin{cases}
            1 & \mbox{if $T(p')=0$},\\
            0 & \mbox{if $T(p')\not=0$}
        \end{cases}
    \]
    where $p'$ is the ROA-prompt $(r'_0,o'_0,a'_0,\ldots,r'_{n+1},o'_{n+1})$
    where $r'_0=0$, each $r'_{i+1}=a_i$,
    each $o'_i=0$, and each
    $a'_i=T(r'_0,o'_0,a'_0,\ldots,r'_i,o'_i)$.
\end{example}

In Example \ref{incentivetoincentivizeexample}, the agent observes
actions, and the agent must act by
choosing rewards for those observed actions.
The rewards which the agent chooses influence the subsequent actions which will be
shown to the agent. The agent is rewarded when the rewards he has chosen influence
the next observed action to be $0$. Unknown to the agent---but an intelligent
self-aware agent should eventually notice the pattern---the way the observed actions
are determined is by giving the agent's chosen rewards to a simulated copy of said agent.
Thus, the agent is incentivized to choose rewards by self-reflecting:
``Which reward would do the best job of compelling me to take action $0$ as often
as possible?'' We might imagine the agent playing a video-game in which he sees himself
in front of a keyboard. The video-game copy-agent types
``$100$'', and then a prompt appears on the screen saying, ``Which reward will you give this
worker for typing $100$ just now?'' The
agent responds by choosing some reward, and sees an animation
of the reward being given to the video-game copy-agent. The video-game copy-agent
then types ``$0$'', and immediately the true agent is rewarded for getting the video-game
copy-agent to type $0$. Then the prompt appears, saying, ``Which reward will you give this
worker for typing $0$ just now?'' And so on
forever\footnote{Example \ref{incentivetoincentivizeexample}
is interesting in that the agent, desiring the clone to take action $0$
as often as possible, is incentivized to choose harsh punishments when the clone takes
nonzero actions.
If punishments are limited to $\mathbb Q$, then the agent faces a dilemma
similar to one in RL cancer treatment applications.
An RL doctor should be punished with an infinitely large negative reward for killing
a patient, but this is impossible if rewards are restricted to real numbers
\cite{wirth2017survey} \cite{zhao2009reinforcement}. Similarly, the agent in Example
\ref{incentivetoincentivizeexample} would probably like to choose infinitely large
rewards, if possible, when its clone takes action $0$. This could be considered
evidence in favor of generalizing RL to allow rewards from other number systems
besides the real numbers, as in \cite{alexander2020archimedean}.}.

\begin{example}
\label{dejavuexample}
    (D\'{e}j\`{a} Vu)
    We define an environment $e$ as follows
    (where $T,p,a$ are as in Example \ref{rewardagentforignoringrewardsexample},
    and with similar non-halting caveats
    as Example \ref{rewardagentforignoringrewardsexample}):
    \begin{align*}
        e(T,\langle\rangle) &= \langle 0,0\rangle\\
        e(T,p\frown a) &= \langle r,0\rangle\\
    \end{align*}
    where
    \[
        r =
        \begin{cases}
            1 & \mbox{if $T(p\frown a\frown p)=a$},\\
            -1 & \mbox{if $T(p\frown a\frown p)\not=a$}.
        \end{cases}
    \]
\end{example}

In Example \ref{dejavuexample},
the agent is incentivized to self-reflect and ask: ``Which action
would I take in order to ensure that I would take that same action if everything
which has happened so far were to repeat itself verbatim?''


Many other interesting examples could be given. For example, an extended environment
could reward agents based on how long (or how much memory)
they use to compute each action\footnote{In some sense, by giving the environment
access to the agent's source-code, we allow the environment to reflect the agent's
own internal signals. Thus, extended environments seem to generalize the idea of
agents modified to manually predict their own internal signals, as in
\cite{sherstan2016introspective}.}. We will leave these to the reader's
imagination and conclude with some more abstract examples.

\section{Some more ambitious examples}

\subsection{Playing in the mirror}

\begin{quote}
    ``I may add that when a few days under nine months old he associated his own name with
    his image in the looking-glass, and when called by name would turn towards the glass
    even when at some distance from it.''---Charles Darwin \cite{darwin1877biographical}
\end{quote}

It has been suggested that the act of recognizing oneself in the mirror is linked
to the development of certain parts of the human psychology
\cite{lacan}. Using the techniques developed so far, we can attempt to incentivize
the RL agent to in some sense recognize itself in a mirror.

\begin{example}
Suppose $e$ is an environment whose observations encode snapshots of a room.
Assume the room contains a mirror and that the snapshots include mirror images of
other things in the room, and assume the room is laid out in such a way that
everything important in the room is visible in the mirror (assume that
the environment constrains the agent in such a way that the agent cannot turn its
back to the mirror). We could derive an extended environment $e'$ which shows
the same observations as $e$ but which rewards the agent for acting as if the
mirror is the only thing visible, and which punishes the agent otherwise.
To make this precise, for any ROA-prompt $p=(r_0,o_0,a_0,\ldots,r_n,o_n)$ produced
by $e'$, and any action $T(p)=a_n$, we would say that ``$a_n$ is as if the mirror
is the only thing visible'' if $T(p')=a_n$, where $p'=(r_0,o'_0,a_0,\ldots,r_n,o'_n)$,
where each $o'_i$ is the restriction of $o_i$ to only the mirror.
\end{example}

To make this even more elaborate, the $o'_i$ in the above example could be further
modified by adding an image of the agent's ``body'' into the mirror. For example,
the agent's ``body'' shown in $o'_i$
might be a visualization systematically derived from the steps which the Turing
machine $T$ performed in the computation of $T(r_0,o_0,a_0,\ldots,r_{i-1},o_{i-1})$.
These steps would not be available to a traditional RL environment, but they are
available to an extended environment because of the inclusion of $T$ itself as an
argument passed to the extended environment.

\subsection{Binocular vision}

Humans seem to consciously perceive a three-dimensional model of their surrounding
world, even though the raw data which we actually receive consists of two two-dimensional
image feeds (one for each eye). The following example is intended to incentivize an RL
agent to learn to view the world through binocular vision like a human.

\begin{example}
\label{binocularexample}
    Suppose $V$ is a video game intended to be played on a virtual-reality headset,
    so at any moment during the game, $V$ produces two snapshots, one for the player's
    left eye, one for the player's right eye. Assume the player is constrained in $V$
    so as never to be able to put their eyes into weird configurations (such as
    the weird configurations in
    \cite{gallagher2020third}): thus, at any moment, the two snapshots $s_1,s_2$
    which $V$ is
    displaying to the player are equivalent to a single 3-D matrix encoding
    the model $m(s_1,s_2)$ which the player is intended to perceive (with cells blanked
    out where the player's vision is obstructed by obstacles). Let $e$ be the extended
    environment whose observations encode pairs $\langle s_1,s_2\rangle$ of snapshots
    displayed by $V$ in response to the player pressing keys encoded by the agent's
    actions. In response to an ROA-play $(r_0,o_0,a_0,\ldots,r_n,o_n,a_n)$ (where
    each $o_i$ encodes $\langle s^i_1,s^i_2\rangle$), let $r_{n+1}=1$ if
    $a_n=T(r_0,o'_0,a_0,\ldots,r_n,o'_n)$, where each $o'_i$ encodes
    $m(s^i_1,s^i_2)$, otherwise let $r_{n+1}=-1$.
\end{example}

In Example \ref{binocularexample}, upon being presented a sequence of pairs of snapshots,
the agent is incentivized to self-reflectively ask, ``Which action would I take if instead
of observing those pairs of snapshots, I observed the 3-D model into which they
stereographically fuse?''

\subsection{Nature and Nurture}

\begin{quote}
    ``If only one soul was created, and all human souls are descended from it,
    who can say that he did not sin when Adam sinned?''---Augustine \cite{augustine1993free}
\end{quote}

How much of our personality is unique to us, and how much is common to all humans?
Would I perform the same actions as you if I were exposed to the exact same stimuli as
you all my life? Probably not, because our bodies are different; but in what sense are
we our bodies, and in what sense are our bodies part of our environment?
The following example is motivated by contemplating the possibility that maybe we all
run the same software on some deep level.

\begin{example}
    (The Crying Baby)
    We define an extended environment $e$ as follows.
    \begin{itemize}
        \item
        Considered as actions taken by an adult
        (and also as observations seen by a baby), let $0$ denote ``feed the baby''
        and let all naturals $>0$ denote ``don't feed the baby''.
        \item
        Considered as actions taken by a baby
        (and also as observations seen by an adult), let $0$ denote ``laugh'' and let
        all naturals $>0$ denote ``cry''.
        \item
        We define a function $s$, which stands for \emph{satiation}, defined on
        ROA-plays, by $s(p)=100+25f(p)-\mbox{len}(p)$ where $f(p)$ is the number of
        times that the action ``feed the baby'' is taken in $p$.
        \item
        Let $e(T,\langle\rangle)=\langle 1,\mbox{``laugh''}\rangle$.
        \item
        For each ROA-play $\langle r_0,o_0,a_0,\ldots,r_n,o_n,a_n\rangle$,
        let
        \[
            e(T,\langle r_0,o_0,a_0,\ldots,r_n,o_n,a_n\rangle)
            =
            \langle r,o\rangle
        \]
        where $r$ and $o$ are defined as follows.
        For each $i=1,\ldots,n$, define
        \begin{align*}
            r'_i &=
                \begin{cases}
                    1 &
                    \mbox{if
                    $50\leq s(\langle r_0,o_0,a_0,\ldots,r_n,o_n,a_n\rangle)\leq 200$,}\\
                    -1 & \mbox{otherwise,}
                \end{cases}\\
            o'_i &= a_i\\
            a'_i &= T(\langle r'_0,o'_0,a'_0,\ldots,r'_i,o'_i\rangle).
        \end{align*}
        Define $o=a'_n$ and
        \[
            r =
            \begin{cases}
                1 & \mbox{if $a'_n=\mbox{``laugh''}$,}\\
                -1 &\mbox{if $a'_n=\mbox{``cry''}$.}
            \end{cases}
        \]
    \end{itemize}
\end{example}

\section{Meta-Examples}

\begin{example}
\label{selfrecognitionexample}
    (Incentivizing Self-recognition)
    Let $p_0,p_1,\ldots$ be a canonical computable enumeration of all non-empty ROA-plays.
    We define an environment $e$ as follows (where $T,p,a$
    are as in Example \ref{rewardagentforignoringrewardsexample}, and with appropriate
    non-halting caveats).
    \begin{align*}
        e(T,\langle\rangle) &= \langle 0, p_0\rangle\\
        e(T,p\frown a) &= \langle r, p_{n}\rangle
    \end{align*}
    where $n=\frac{\mbox{len}(p\frown a)}{3}$ is the number of actions in $p\frown a$ and where
    \[
        r =
        \begin{cases}
            1 &\mbox{if $a>0$ and $a'=T(p')$,}\\
            1 &\mbox{if $a=0$ and $a'\not=T(p')$,}\\
            0 &\mbox{otherwise}
        \end{cases}
    \]
    where $p_{n-1}=p'\frown a'$.
\end{example}

In Example \ref{selfrecognitionexample}, the agent is systematically
shown all non-empty ROA-plays, and for
each ROA-play, the agent either types ``Looks like me'' (any action $>0$)
or ``Doesn't look like me'' ($0$). When shown the non-empty ROA-play
$p'\frown a'$, the agent is rewarded if and only if the agent
correctly determines whether or not $a'$
is the action the agent itself would take in response to $p'$.
Thus, the agent is incentivized to self-reflect in order to ask, ``If I experienced the
observations and rewards in that ROA-prompt, would I act that way?''

The following example is partly motivated by \cite{yampolskiy2012ai}.

\begin{example}
\label{otheraspectsexample}
    (Recognizing other aspects of oneself)
    All the below environments are similar to Example \ref{selfrecognitionexample},
    and we describe them informally to avoid technical details.
    \begin{itemize}
        \item
        (Supervised learning)
        Assume there is a canonical, computable function $f$ which transforms
        each RL agent $A$ into a supervised learning agent $f(A)$. By a \emph{supervised
        learning trial} we mean quadruple $\langle L,T,I,p\rangle$ where $L$ is a finite set
        of labels, $T$ is a sequence of images with labels from $L$ (a \emph{training set}),
        $I$ is an unlabeled image, and $p:L\to \mathbb Q\cap [0,1]$ is a function
        assigning to each label $\ell\in L$ a probability that $\ell$ is the correct label
        for $I$. We define an extended environment as follows.
        The agent $A$ is sytematically shown all supervised learning trials and must
        take action $>0$ (``Looks like me'') or $0$ (``Doesn't look like me''), and is
        rewarded or punished depending whether or not $f(A)$ would
        output $p$ in response to $I$ after being trained with $T$.
        \item
        (Unsupervised learning)
        Assume there is a canonical, computable function $g$ which transforms each RL
        agent $A$ into an unsupervised learning agent $g(A)$.
        By an \emph{unsupervised learning trial} we mean a triple
        $\langle n,D,C\rangle$ where $n$ is a positive integer, $D\subseteq \mathbb Q^n$
        is a finite set of $n$-dimensional points with rational coordinates, and $C$
        is a clustering of $D$.
        We define an extended environment as follows. The agent $A$ is systematically
        shown all unsupervised learning trials and must take action $>0$ (``Looks like me'')
        or $0$ (``Doesn't look like me''), and is rewarded or punished depending
        whether or not $g(A)$ would cluster $D$ into clustering $C$.
        \item
        (The Turing Test)
        Assume there is a canonical, computable function $h$ which transforms each RL
        agent $A$ into an English-speaking chatbot $h(A)$.
        By a \emph{chatbot trial} we mean a sequence of strings of English characters.
        We define an extended environment as follows. The agent $A$ is systematically
        shown all chatbot trials and must take action $>0$ (``Looks like me'') or $0$
        (``Doesn't look like me''), and is rewarded or punished accordingly depending
        whether or not even-numbered strings in the chatbot trial are what $h(A)$ would
        say in response to the user saying the odd-numbered strings.
        \item
        (Adversarial sequence prediction) Similar to the above environments, assuming
        a canonical computable function which transforms each RL agent into a predictor
        in the game of adversarial sequence prediction \cite{hibbard2008adversarial}
        \cite{hibbard}.
        \item
        (Mechanical knowing agent) Assume there is a canonical, computable function
        $i$ which transforms each RL agent $A$ into a code $i(A)$ of a computably
        enumerable set of sentences in the language of Epistemic Arithmetic
        \cite{shapiro}; $i(A)$ is thought of as a mechanical knowing agent
        \cite{carlson}. We define an extended environment as follows. The agent $A$
        is systematically shown all sentences in the language of Epistemic Arithmetic,
        with repetition, in such a way that each sentence is shown infinitely often.
        Upon being shown sentence $\phi$ for the $n$th time, the agent must take action
        $>0$ (``I know $\phi$ is true'')
        or $0$ (``I'm not sure if $\phi$ is true'')
        and is rewarded if and only if $\phi$ is
        enumerated by $i(A)$ in $\leq n$ steps of computation.
    \end{itemize}
\end{example}

The extended environments in Example \ref{otheraspectsexample} incentivize the agent
to self-reflect, asking itself questions like ``Does that look like how I would
classify that image, given that training set?'' or ``Does that look like how I would
cluster that set of points?'' or ``Does that conversation participant say the same things
I would say?'' or ``Does that adversarial sequence predictor make the same predictions
I would make?'' or ``Does that mechanical knowing agent know the same things I know?''

\bibliographystyle{splncs04}
\bibliography{intro}

\end{document}